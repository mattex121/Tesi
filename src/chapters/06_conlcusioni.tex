\chapter{Conclusioni}
\label{chp:Conclusioni}
Data la necessita sempre più impellente di virare la produzione energetica verso tecnologie più sostenibili per l'ambiente, sono stati presi in analisi nello specifico gli impianti rinnovabili. Da questa ho potuto osservare come spesso si faccia riferimento solo all'inquinamento diretto prodotto da un impianto senza però considerare tutti i fattori di inquinamento indiretti.
Ci troviamo spesso davanti ad impianti che hanno un alto impatto paesaggistico e per la fauna generando impedimenti architettonici o variazioni dell'ambiente naturale tali da alterarne le proprietà.\\
Un ulteriore fattore che rende molto difficile questa transizione ecologica è l'impossibilità di prevedere e controllare la produzioni delle centrali a causa dell'irregolarità dei fenomeni meteorologici. Fenomeni che non essendo costanti e distribuiti in modo uniforme sulla terra rendono solo un numero ridotto di luoghi adatti ad ospitare uno specifico impianto.\\
Stiamo vivendo un momento storico nel quale il meteo è sempre più irregolare con, ad esempio, forti fenomeni piovosi e lunghi periodi di siccità. Caratteristiche che sono totalmente inadeguate in un mondo in cui la richiesta energetica è costante ed in crescita.\\
Nel capitolo finale sono state prese in analisi le caratteristiche di un impianto fotovoltaico ponendolo sempre nelle condizioni di lavoro ottimali rispetto al suo posizionamento geografico. Spesso ci si trova, per ridurre fenomeni di larga occupazione del suolo, ad installare il fotovoltaico su superfici non ideali quali ad esempio tetti di edifici dell'agglomerato urbano.
Lo scopo di queste installazioni è quello di cercare di far avvicinare la domanda di energia con la produzione andando a sfruttare spazi che in caso contrario sarebbero stati inutilizzati. Risulta facilmente intuibile come l'installazione superfici esistenti risulti difficoltosa od a volte addirittura poco conveniente data la scarsa resa di piani non correttamente inclinati ed orientati.\\
Ci troviamo di fronte ad una situazione quindi molto difficile in quanto è assolutamente necessario procedere con una veloce decarbonizzazione per ridurre il costante incremento della temperatura globale e mitigare i cambiamenti climatici. Per perseguire questa strada però difficilmente sarà possibile utilizzare solo fonti rinnovabili per come sono tecnologicamente concepite nel periodo odierno.